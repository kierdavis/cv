\heading{Projects \& Achievements}

\vspace{-1.2em}

\begin{longtable}{L!{\sep}R}
  \rctline{Oct 2017--present}{
    Developing a system for interfacing a modern computer to a machine for producing punched paper tape, which will ultimately be used at the \nmoc{} to facilate use of their \replica{} of the historic EDSAC computer. This project is an opportunity to demonstrate my skills of system/electronic/software design and my abilities to communicate and meet schedules, as well as forming a major part of my degree course.
  }

  \rctline{Aug 2017}{
    Volunteered to help organise the annual \summerschool{} at the university for the second year running, in which teams of 14--15 year olds complete a competitive engineering project with the aid of expert support and mentoring.
  }

  \rctline{Mar 2017}{
    Designed and constructed a quadcopter from scratch as part of a team design project. I led the development of the embedded electronics and control software, and exercised my organisational and technical skills in a very time-constrained environment.
  }

  \rctline{Oct 2016--May 2017}{
    Developed the control logic for a simple stopwatch on an ASIC for a university group project. The design was then manufactured, following which testing was performed on the finished ICs. The extensive verification performed before the design was sent off resulted in the stopwatch controller working flawlessly.
  }

  \rctline{Oct 2016}{
    Awarded the Zepler prize by the university, for obtaining the highest grade in first year out of all students on my course.
  }

  \rctline{Apr 2016}{
    Competed in the Inter-ACE cyber security competition as part of a team representing the university. The competition consisted of solving a variety of reverse engineering, binary exploitation and cryptography challenges in a race against the clock. We placed third out of ten teams.
  }

  % \rctline{Jan 2016}{
  %   Built and tested a boost converter power supply for a design exercise as part of my degree course. The project involved running detailed simulations of the circuit, constructing the circuit on a protoboard, writing embedded control software in \texttt{C}, and testing and tuning the circuit to produce optimal results.
  % }

  % \rctline{Aug 2015}{
  %   Began working on a hobby project to build a minimal working 8-bit processor out of discrete logic ICs, as inspired by a number of similar projects that I have come across online. Finding a balance between ease of implementation and ease of use has been a challenge and the processor has been redesigned a number of times because of this.
  % }

  \rctline{Nov 2014--Apr 2015}{
    Competed in the 2015 \studentrobotics{} competition as part of a team of approximately eight people from my school, in which we were required to design and build an autonomous robot capable of playing a game of Capture the Flag. Despite having to overcome a number of challenges during the development of the robot, we managed to reach the semi-finals of the competition and place sixth out of fifty teams overall.
  }

  % \rctline{Mar 2014}{
  %   Completed a software system for planning meals for the week as part of my A-level Computing course. The task involved designing, implementing and testing a system for a real client.
  % }

  % \rctline{Apr 2013}{
  %   Submitted my coursework for my GCSE Electronics qualification, which involved designing and constructing a small household burglar alarm system. I was responsible for the entire product design including devising the circuit and producing a custom printed circuit board.
  % }

  \rctline{Aug 2012}{
    Participated in the Young Rewired State week-long hackathon, where we built a web application utilising open government data to compare living conditions between different locations in the UK. My team and I came first in the \emph{Best Example of Coding} category, winning a 3D printer kit.
  }

  % \rctline{Mar 2011}{
  %   Entered the national \popmathsquiz{} hosted by Sheffield Hallam University as part of a team from my school, placing first among roughly forty teams.
  % }
\end{longtable}

\vspace{.2em}
