\heading{Skills and Experience}

\begin{longtable}{L!{\sep}R}
    Aug 2015 & Begin working on a hobby project to build a small working 8-bit processor out of discrete logic ICs, as inspired by a number of similar projects. The hardware designs and software code are open source and are available online.\footnote{\texttt{https://github.com/kierdavis/k12}} \vspace{1.2em} \\

    Apr 2015 & Competed in the 2015 Student Robotics competition as part of a team of around 8 people from my school, in which we were required to design and build an autonomous robot capable of playing a game of Capture the Flag. Despite having to overcome a number of challenges during the development of the robot, we managed to reach the semi-finals of the competition and place sixth overall. \vspace{1.2em} \\

    Mar 2014 & Completed a software system for planning meals for the week as part of my A-level Computing course. The task involved designing, implementing and testing a system for a real client. \vspace{1.2em} \\

    %Jun 2013 & Developed a number of plugins for Minecraft, an online multiplayer game. They were written in Java and are now in use by dedicated game servers across the world. \vspace{1.2em} \\

    Feb 2013 & Engaged in a week-long work experience placement at a local software company, \mbox{Pengower Ltd}. I assisted in the development and testing of their platform for business applications, which was written in \texttt{C\#}. \vspace{1.2em} \\

    Apr 2013 & Submitted my coursework for my GCSE Electronics qualification, which involved designing and constructing a small household burglar alarm system. I was responsible for the entire product design including devising the circuit and producing a custom PCB. \vspace{1.2em} \\

    Aug 2012 & Participated in the Young Rewired State week-long programming competition. My team and I came first in the \emph{Best Example of Coding} category and won a 3D printer, which we then spent a few months building. \vspace{1.2em} \\

    Jul 2012 & Developed a software library for manipulating RDF, an open format for metadata. This project used the Go programming language, and its source code is made available on Github. \vspace{1.2em} \\

    Mar 2011 & Entered the national Pop Maths Quiz hosted by Sheffield Hallam University as part of a team from my school, placing first among over a hundred teams. \\
\end{longtable}
